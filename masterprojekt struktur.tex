\begin{document}



\section{Einleitung}

\section{Problemstellung}
    
    \subsection{Thematischer Schwerpunkt}

\section{Zielsetzung}

\section{Vergleich unterschiedlicher Ansätze zur Umsetzung}%

    \subsection{Mögliche Betriebsarten des Systems}

    \subsection{Einführung in große Sprachmodelle (LLM)}

    \subsubsection{Grundlagen der Sprachmodelle}

    \subsubsection{Architektur der Sprachmodelle}

    \subsubsection{Anwendungen der Sprachmodelle}

\section{Formale Evaluation der Systeme}%

\section{Anpassung der Systeme auf den Anwendungszweck}%

    \subsection{Finetuning}

        \subsubsection{Finetuning Prozess}%

        \subsubsection{Erklärung der verwendeten Techniken}%

        \subsubsection{Erklärung des Training-Prozesses}%

        \subsubsection{Durchführung der Tests}%

        \subsubsection{Analyse der gegebenen Antworten}%

    \subsection{Retrieval Augmented Generation (RAG)}%

        \subsubsection{RAG-Algorithmus}%

        \subsubsection{Durchführung der Tests}%

        \subsubsection{Analyse der gegebenen Antworten}%

\section{Zukünftige Arbeiten}%

    \subsection{Verbesserung des aktuellen Ansatzes}%

    \subsection{Kombinierte Ansätze}

    \subsection{Hawki}%

\section{Zusammenfassung}%


\begin{center}
    \section*{Abkürzungsverzeichnis}
\end{center}

\begin{center}
    \begin{acronym}[FAU] % Optionale Argument: längste Abkürzung für richtige Ausrichtung
        \acro{FAU}{Friedrich-Alexander-Universität Erlangen-Nürnberg}
        \acro{LLM}{Large Language Model}
        \acro{GPU}{Graphics Processing Unit}
        \acro{CPU}{Central Processing Unit}
        \acro{QLoRA}{Quantized Low-Rank Adaptation}
        \acro{LoRA}{Low-Rank Adaptation}
        \acro{PEFT}{Parameter-Efficient Finetuning}
        \acro{RAG}{Retrieval Augmented Generation}
        \acro{CUDA}{Compute Unified Device Architecture}
        \acro{ROCm}{Radeon Open Compute}
        \acro{VRAM}{Video Random Access Memory}
        \acro{SFT}{Supervised Finetuning}
        \acro{SFTTrainer}{Supervised Finetuning-Trainer}
        \acro{MLC LLM}{Machine Learning Compilation}
        \acro{RAM}{Random Access Memory}
        \acro{iGPU}{Integrated Graphics Processing Unit}
        \acro{dGPU}{Dedicated Graphics Processing Unit}
        \acro{JSONL}{JavaScript Object Notation Lines}
        \acro{DP}{DataParallel}
        \acro{DDP}{DistributedDataParallel}
        \acro{GIL}{Global Interpreter Lock}
        \acro{HPC}{High Performance Computing}
        \acro{NPP}{Naive Pipeline Parallelism}
        \acro{GB}{Gigabyte}
        \acro{MHz}{Megahertz}
        \acro{API}{Application Programming Interface}
        \acro{FP16}{Half-precision floating-point format}
        \acro{FP32}{Single-precision floating-point format}
        \acro{FP64}{Double-precision floating-point format}
        \acro{BF16}{bfloat16 floating-point format}
        % \acro{}{}

    \end{acronym}
\end{center}

\newpage
\makebib%

% TODO urldate
% TODO hat jedes Diagramm, Grafik auch eine Unterschrift?
% TODO Alle Probleme fixen
% TODO noch Seitenumbrüche einbauen

\end{document}

