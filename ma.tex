Gute Links:
\url{https://www.promptfoo.dev/blog/how-to-jailbreak-llms/}
\url{https://www.br.de/nachrichten/meldung/russland-manipuliert-laut-studie-westliche-chatbots-fuer-propaganda,300710b13}

% chktex-file 44
% chktex-file 8
% chktex-file 24
% chktex-file 13
% chktex-file 29
% chktex-file 31

\documentclass[german,report]{i1thesis}

\usepackage{xcolor,colortbl}
\usepackage{amsmath}
\usepackage{biblatex}
\usepackage{pgfplots}
\usepackage{acronym}


\def\theauthor{Fabian Berger, B. Sc.}
\def\theadvisor{Prof.\ Dr.-Ing. Michael Tielemann}

% Es ist möglich mehrere Autoren, Betreuer oder Prüfer einzutragen:
%\def\theadvisor{Some guy, M.Sc., \\ & Some other guy, M.Sc}s
% Wenn gendergerechte Sprache gewünscht ist bzw. bei Autor/Prüfer ein Plural 
% gebraucht wird, lässt sich das Keyword so anpassen:
%\def\textadvisor{Betreuende}

\def\theexaminer{Prof.\ Dr.-Ing. Michael Tielemann}
\def\thetitle{Evaluierung eines Chatbots zur Verbesserung der Informationssicherheit an der FAU\@ Lokale Modelle, Plattformvergleich und Integrationsmöglichkeiten}
\def\theplace{Erlangen}
\def\thedate{\today}

% Diese Befehle ändern das jeweilige Keyword auf der Hauptseite
%\def\germanAuthor{Autor}
%\def\germanBetreuer{Betreuer}
%\def\germanPruefer{Prüfer}


\begin{document}

\makethetitle%

\maketoc%

\newpage
\section{Einleitung}
\label{sec:einleitung}


\section{Problemstellung}
\label{sec:problemstellung}


\subsection{Thematischer Schwerpunkt}
\label{subsec:thematischer-schwerpunkt}


\section{Zielsetzung}
\label{sec:zielsetzung}

%----------------------------------

% laut bdsem:
% verwandte arbeiten
% Konzept/auswahlmethodik
% grundlegende kompressionstechniken
% \dots
% Zusammenführung der Ergebnisse
% Evaluation
% SCHLUSS
% Zusammenfassung
% Zukünftige Arbeiten / Ausblick

\section{Vergleich unterschiedlicher Ansätze zur Umsetzung}
\label{sec:vergleich}

\subsection{Mögliche Betriebsarten des Systems}




\subsection{Einführung in große Sprachmodelle (LLM)}

\subsubsection{Grundlagen der Sprachmodelle}


\subsubsection{Architektur der Sprachmodelle}


\subsubsection{Anwendungen der Sprachmodelle}



\section{Formale Evaluation der Systeme}%
\label{sec:formale-evaluation}




\section{Anpassung der Systeme auf den Anwendungszweck}%
\label{sec:anpassungen-systeme}
\subsection{Finetuning}%
\label{subsec:Finetuning-cloud}




\subsubsection{Finetuning Prozess}%
\label{subsec:Finetuning-prozess}



\subsubsection{Erklärung der verwendeten Techniken}%



\subsubsection{Erklärung des Training-Prozesses}%
\label{subsec:training-prozess}



\subsubsection{Durchführung der Tests}%
\label{subsec:durchfuehrung-der-tests}

\subsubsection{Analyse der gegebenen Antworten}%
\label{subsec:Analyse der gegebenen Antworten}

\subsection{Retrieval Augmented Generation (RAG)}%
\label{subsec:rag}




\subsubsection{RAG-Algorithmus}%
\label{subsec:rag-alg}

\subsubsection{Durchführung der Tests}%
\label{subsec:durchfuehrung-der-tests-rag}


\subsubsection{Analyse der gegebenen Antworten}%
\label{subsec:analyse-der-gegebenen-antworten-rag}



\section{Zukünftige Arbeiten}%
\label{sec:zukuenftige-arbeiten}

\subsection{Verbesserung des aktuellen Ansatzes}%
\label{subsec:verb}

\subsection{Kombinierte Ansätze}


\subsection{Hawki}%
\label{subsec:hawki}

\section{Zusammenfassung}%
\label{sec:zusammenfassung}


\newpage

\begin{center}
    \section*{Abkürzungsverzeichnis}
\end{center}

\begin{center}
    \begin{acronym}[FAU] % Optionale Argument: längste Abkürzung für richtige Ausrichtung
        \acro{FAU}{Friedrich-Alexander-Universität Erlangen-Nürnberg}
        \acro{LLM}{Large Language Model}
        \acro{GPU}{Graphics Processing Unit}
        \acro{CPU}{Central Processing Unit}
        \acro{QLoRA}{Quantized Low-Rank Adaptation}
        \acro{LoRA}{Low-Rank Adaptation}
        \acro{PEFT}{Parameter-Efficient Finetuning}
        \acro{RAG}{Retrieval Augmented Generation}
        \acro{CUDA}{Compute Unified Device Architecture}
        \acro{ROCm}{Radeon Open Compute}
        \acro{VRAM}{Video Random Access Memory}
        \acro{SFT}{Supervised Finetuning}
        \acro{SFTTrainer}{Supervised Finetuning-Trainer}
        \acro{MLC LLM}{Machine Learning Compilation}
        \acro{RAM}{Random Access Memory}
        \acro{iGPU}{Integrated Graphics Processing Unit}
        \acro{dGPU}{Dedicated Graphics Processing Unit}
        \acro{JSONL}{JavaScript Object Notation Lines}
        \acro{DP}{DataParallel}
        \acro{DDP}{DistributedDataParallel}
        \acro{GIL}{Global Interpreter Lock}
        \acro{HPC}{High Performance Computing}
        \acro{NPP}{Naive Pipeline Parallelism}
        \acro{GB}{Gigabyte}
        \acro{MHz}{Megahertz}
        \acro{API}{Application Programming Interface}
        \acro{FP16}{Half-precision floating-point format}
        \acro{FP32}{Single-precision floating-point format}
        \acro{FP64}{Double-precision floating-point format}
        \acro{BF16}{bfloat16 floating-point format}
        % \acro{}{}


    \end{acronym}
\end{center}

\newpage
\makebib%

% TODO urldate
% TODO hat jedes Diagramm, Grafik auch eine Unterschrift?
% TODO Alle Probleme fixen
% TODO noch Seitenumbrüche einbauen

\end{document}

